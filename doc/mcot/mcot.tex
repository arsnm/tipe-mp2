\documentclass[a4paper, 11pt]{article}

\author{Arsène MALLET}

\usepackage[french]{babel}
\usepackage[utf8]{inputenc}
\usepackage[T1] {fontenc}
\usepackage[backend=biber, maxbibnames=5, style=numeric, sorting=none]{biblatex}
\usepackage{csquotes}
\DeclareCiteCommand{\supercite}[\mkbibsuperscript]
  {\iffieldundef{prenote}
     {}
     {\BibliographyWarning{Ignoring prenote argument}}%
   \iffieldundef{postnote}
     {}
     {\BibliographyWarning{Ignoring postnote argument}}}
  {\usebibmacro{citeindex}%
   \textbf{\bibopenbracket\usebibmacro{cite}\bibclosebracket}}
  {\supercitedelim}
  {}

\addbibresource{refs.bib}

\let\cite=\supercite

\title{Modélisation de vie artificielle à l'aide d'automates cellulaires continues}

\begin{document}
    
\begin{center}
{\textbf {Modélisation de vie artificielle à l'aide d'automates cellulaires continues}}
\end{center}

Lorsque j'étais plus jeune, je m'étais amusé à jouer au "Jeu de la Vie", une simulation
de cellules facile à prendre en main, mais aux possibilités gigantesques. Il m'a donc
paru intéressant de pousser les recherches sur la vie artificielle et les automates
cellulaires.

Bien qu'il n'y est pas de "joueur" à proprement parler dans ce type de simulation, 
l'étude des paramètres de simulation permettant l'observation de phénomènes intéressant
est une tâche complexe qui peut être considéré comme un jeu. C'est dans ce cadre que 
s'inscrit mon TIPE dans le thème de l'année.

\vspace{5mm}

\section*{Professeur encadrant du candidat}
Q. Fortier

\section*{Positionnement th\'ematique}
\begin{itemize}
    \item INFORMATIQUE(\textit{Informatique Pratique})
    \item MATHEMATIQUES(\textit{Math\'ematiques Appliqu\'ees})
\end{itemize}

\section*{Mots-cl\'es}

\begin{tabular}{l l} 
    (\textit{français}) & (\textit{anglais}) \\ \hline
    Automate Cellulaire & Cellular Automaton \\
    Vie artificielle & Artificial Life \\
    Sensorimoteur & Sensorimotor \\
    Apprentissage Automatique & Machine Learning \\
    Fonction de Croissance & Growth Mapping \\
    Noyau (convolutif) & Kernel \\ \hline
    \end{tabular}

\section*{Bibliographie comment\'ee}

La vie artificielle est un domaine de recherche faisant appel à des notions 
mathématiques, informatiques et biologiques. En s'inspirant de systèmes vivants, la
recherche sur la vie artificielle est divisée en trois axes \cite{BEDAU2003505}. La 
ALife (A[rtificial ]Life) "soft", qui consiste en la simulation informatique de
comportements semblables à ceux de la vie ; l'axe "hard" qui produit quant à lui des
objets matérielles, et enfin la ALife "wet", qui synthétise des systèmes vivants à
partir de substances biochimiques.

Lorsqu'il est question de vie artificielle "soft", ce sur quoi nous allons nous
focaliser, il existe encore divers sous-branches, comme par exemple les réseaux de
neuronnes, les algorithmes génétiques ou les automates cellulaires 
\cite{komosinski2009artificial}.

L'automate cellulaire le plus connu est surement le "Jeu de la vie", mis au point par
le mathématicien anglais Jonh Conway en 1970 \cite{izhikevich2015game}. C'est un jeu à
zéro joueur, il est appelé ainsi car il n'est défini que par son état initial, fixé par
le joueur (joueur qui n'intervient plus par la suite). Ce jeu consiste en la vie de 
cellules sur une grille, qui naissent, meurent et survivent en fonction des cellules 
vivantes de leur voisinnage.
50 ans plus tard, en 2019, le chercheur B. W. Chan publie une généralisation continue du
modèle discret de Conway, qu'il nomme Lenia \cite{Chan_2019}.

Cette généralisation fait appel à des noyaux convolutifs et des fonctions de croissance
\cite{Chan_2019}, permettant, tout comme le Jeu de la Vie, de connaître l'évolution des
cellules au cours du temps et de l'espace.

Mais les configurations initiales de Lenia sont infinies, ainsi il convient de les
choisir au mieux si l'on veut observer des phénomènes remarquables. Mais comment bien
choisir ces attributs ? C'est ici que l'apprentissage automatique s'avert utile.
Par des moyens de machine learning, il est possible de chercher des configurations
stables, résistantes à certaines perturbations, capablent de sortir d'un labyrinthe, ou
encore de se déplacer vers des zones plus favorables à la survie des cellules 
\cite{hamon:hal-03519319} \cite{plantec2023flowlenia}.

Il est également possible de faire des analogies entre notre système de cellules et un
réseau de neuronnes, permettant ainsi d'appliquer des algorithmes de rétro-propagation,
qui permettent d'orienter nos cellules vers le comportement souhaité 
\cite{hamon:hal-03519319}.

Finalement, les méthodes permettant l'observation de phénomènes macroscopiques
remarquables à partir de règles élémentaires sont nombreuses, il conviendra donc
d'explorer ces différentes voies afin de simuler une forme de vie grâce à ce modèle.


\section*{Probl\'ematique Retenue}

Comment optimiser les attributs d'automates cellulaires continues afin de créer des
structures au comportement macroscopiques remarquables ?

\section*{Objectif du TIPE}

\begin{itemize}
    \item Implémentation d'une simulation de vie artificielle au moyen d'automates
        cellulaires continues
    \item Recherche de paramètres permettant des comportement macroscopiques
        remarquables
    \item Chercher à combiner les stuctures afin d'obtenir une certaine forme de
        groupement de cellules intelligente ou adaptative.
\end{itemize}

\printbibliography[title=Références bibliographiques] \end{document}
\end{document}
