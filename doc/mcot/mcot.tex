\documentclass[a4paper, 11pt]{article}

\author{Arsène MALLET}

\usepackage[french]{babel}
\usepackage[utf8]{inputenc}
\usepackage[T1] {fontenc}
\usepackage[backend=biber, maxbibnames=5, style=numeric, sorting=none]{biblatex}
\usepackage{csquotes}
\DeclareCiteCommand{\supercite}[\mkbibsuperscript]
  {\iffieldundef{prenote}
     {}
     {\BibliographyWarning{Ignoring prenote argument}}%
   \iffieldundef{postnote}
     {}
     {\BibliographyWarning{Ignoring postnote argument}}}
  {\usebibmacro{citeindex}%
   \textbf{\bibopenbracket\usebibmacro{cite}\bibclosebracket}}
  {\supercitedelim}
  {}

\addbibresource{refs.bib}

\let\cite=\supercite

\title{Modélisation de vie artificielle à l'aide d'automates cellulaires continues}

\begin{document}
    
\begin{center}
{\textbf {Modélisation de vie artificielle à l'aide d'automates cellulaires continues}}
\end{center}

Lorsque j'étais plus jeune, je m'étais amusé à jouer au "Jeu de la Vie", une simulation
de cellules facile à prendre en main, mais aux possibilités gigantesques. Il m'a donc
paru intéressant de pousser les recherches sur la vie artificielle et les automates
cellulaires.

Bien qu'il n'y est pas de "joueur" à proprement parler dans ce type de simulation, 
l'étude des paramètres de simulation permettant l'observation de phénomènes intéressant
est une tâche complexe qui peut être considéré comme un jeu. C'est dans ce cadre que 
s'inscrit mon TIPE dans le thème de l'année.

\vspace{5mm}

\section*{Professeur encadrant du candidat}
Q. Fortier

\section*{Positionnement th\'ematique}
\begin{itemize}
    \item INFORMATIQUE(\textit{Informatique Pratique})
    \item MATHEMATIQUES(\textit{Math\'ematiques Appliqu\'ees})
\end{itemize}

\section*{Mots-cl\'es}

\begin{tabular}{l l} 
    (\textit{français}) & (\textit{anglais}) \\ \hline
    Automate Cellulaire & Cellular Automaton \\
    Vie artificielle & Artificial Life \\
    Sensorimoteur & Sensorimotor \\
    Apprentissage Automatique & Machine Learning \\
    Fonction de Croissance & Growth Mapping \\
    Noyau (convolutif) & Kernel \\ \hline
    \end{tabular}

\section*{Bibliographie comment\'ee}

La vie artificielle est un domaine de recherche faisant appele à des notions 
mathématiques, informatique et biologique. En s'inspirant de systèmes vivants, la
recherche sur la vie artificielle est divisé en trois axes \cite{artLife}. La ALife
(artifical life / vie artificielle) "soft", qui consiste en la simulation de
comportements semblables à ceux de la vie ; l'axe "hard" qui produit quant à elle des
objets matérielles, et enfin la ALife "wet", qui synthétise des systèmes vivants à
partir de substances biochimiques.

Lorsqu'il est question de vie artificielle "soft", ce sur quoi nous allons nous
focaliser, il existe encore divers sous-branches, comme par exemple les réseaux de
neuronnes, ou les automates cellulaires.

L'automate cellulaire le plus connu est surement le "Jeu de la vie", mis au point par
le mathématicien anglais Jonh Conway en 1970. C'est un jeu a zero joueur, il est appelé
ainsi car il n'est défini que par son était initial, fixé par le joueur, qui
n'intervient plus par la suite. 50 ans plus tard, en 2019, le


\section*{Probl\'ematique Retenue}

Comment optimiser les attributs d'automates cellulaires continues afin de créer des
structures au comportement macroscopiques remarquables ?

\section*{Objectif du TIPE}

\begin{enumerate}
    \item Implémentation d'une simulation de vie artificielle au moyen d'automates
        cellulaires continues
    \item Recherche de paramètres permettant des comportement macroscopiques
        remarquables
    \item Comparaison du comportement du modèle avec des phénomènes observables dans le
        réel
\end{enumerate}

\printbibliography[title=Références bibliographiques] \end{document}
\end{document}
