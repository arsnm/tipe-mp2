\documentclass[a4paper, 11pt]{article}

\author{Arsène MALLET}

\usepackage[french]{babel}
\usepackage[utf8]{inputenc}
\usepackage[T1] {fontenc}
\usepackage[backend=biber, maxbibnames=5, style=numeric, sorting=none]{biblatex}
\usepackage{csquotes}
\DeclareCiteCommand{\supercite}[\mkbibsuperscript]
  {\iffieldundef{prenote}
     {}
     {\BibliographyWarning{Ignoring prenote argument}}%
   \iffieldundef{postnote}
     {}
     {\BibliographyWarning{Ignoring postnote argument}}}
  {\usebibmacro{citeindex}%
   \textbf{\bibopenbracket\usebibmacro{cite}\bibclosebracket}}
  {\supercitedelim}
  {}

\addbibresource{refs.bib}

\let\cite=\supercite

\title{Les réseaux de neurones et la reconnaissance de mouvements}

\begin{document}
    
\begin{center}
    {\textbf {\LARGE Les réseaux de neurones et la reconnaissance de mouvements}}
\end{center}

\vspace{5mm}

Dans un contexte actuel où "l'intelligence artificielle" et le "machine learning"
(apprentissage automatique) font les unes de nombreux médias, le sujet choisi
propose d'étudier et d'implémenter un algorithme de réseau de neurones, populaire
parmi les algorithmes d'apprentissage, afin de l'entrainer et de l'optimiser pour qu'il
puisse reconnaitre des mouvements.

Comment fonctionne un réseau de neurones ? Quels sont les méthodes d'entraimenents ?
Existe-t-il des méthodes spécifiques adaptés à la reconnaissance de mouvements ? 
Le but du projet de comprendre et de manipuler les différents paramètres d'un réseau
de neurones dans le but qu'il réalise au mieux les tâches requises.

\section*{Professeur encadrant du candidat}
Q. Fortier

\section*{Positionnement th\'ematique}
\begin{itemize}
    \item INFORMATIQUE(\textit{Informatique Pratique})
    \item MATHEMATIQUES(\textit{Math\'ematiques Appliqu\'ees})
\end{itemize}

\section*{Mots-cl\'es}

\begin{tabular}{l l} 
    (\textit{français}) & (\textit{anglais}) \\ \hline
     & \\
    Réseau de Neurones & Neural Network \\
    Rétropropagation & Backpropagation \\
    Sigmoïde & Sigmoid Function \\
    Couche & Layer \\
    Algorithme du Gradient Stochastique & Stochastic Gradient Descent
    \end{tabular}

\section*{Bibliographie comment\'ee}

Dans le vaste monde de l'apprentissage automatique, il existe un très grand nombre
d'algorithme, chacun ayant des avantages et des inconvénients. Il convient donc de
choisir lequel de ces algorithmes est le plus approprié á la reconnaissance de
mouvements.
Une de ces solutions est le réseau de neurones convolutionelles, grandement efficace
pour la reconnaissance d'images et de vidéos /\cite{cnn}

\section*{Probl\'ematique Retenue}
Il s'agit d'étudier et d'implémenter différents types de réseaux de neuronnes afin de
tester leur efficacité et leurs limitations lorsqu'il est question de reconnaître 
divers mouvements.

\section*{Objectif du TIPE}
\begin{enumerate}
    L'objectif principal du TIPE est l'implémentation d'un réseau de neurones
    permettant la reconnaissance de mouvements simples. Dans l'idéal, un second objectif
    serait de réussir à ce que le modèle arrive à classer des enchainements de mouvements
    simples ou des mouvements plus complexes.
\end{enumerate}


\printbibliography[title=Références bibliographiques]
\end{document}
