\documentclass[a4paper, 11pt]{article}

\author{Arsène MALLET}

\usepackage[french]{babel}
\usepackage[utf8]{inputenc}
\usepackage[T1] {fontenc}

\title{Mod\'elisation de vie artificielle à l'aide d'automates cellulaires continus}

\begin{document}
    
\begin{center}
    {\textbf {\LARGE DOT - Mod\'elisation de vie artificielle à l'aide d'automates cellulaires continus}} \\
    \vspace{3mm}
    {\small Arsène MALLET}
\end{center}

\vspace{5mm}

\begin{itemize}
    \item $\bigl[$Septembre 2023 - Choix initial du sujet sur les r\'eseaux de neuronnes et la reconnaissance de mouvement dans le sport.$\bigr]$
    \item $\bigl[$Septembre-Novembre 2023 - Renseigment sur les r\'eseaux de neuronnes, apparitions de difficult\'es d'impl\'ementation.$\bigr]$
    \item $\bigl[$D\'ecembre 2023 - D\'ecouverte des automates cellulaires Lenia et analogie de leur "entrainement" avec des r\'eseaux de neuronnes -> R\'eorientation du sujet. $\bigr]$
    \item $\bigl[$Janvier-Mars 2024 -  Impl\'ementation de Lenia, en partant du "Jeu de la Vie" de Conway, jusqu'\`a sa version \'etendue, avec plusieurs noyaux de convolutions et plusieurs cannaux.$\bigr]$
    \item $\bigl[$Avril-Mai 2024 - Impl\'ementation de la partie "apprentissage automatique" des automates cellulaires. Recherche des param\`etres permettant l'observation de ph\'enom\`emes remarquables. Exploitation des donn\'ees.$\bigr]$
\end{itemize}

\end{document}
