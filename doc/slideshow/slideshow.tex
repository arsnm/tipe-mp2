\documentclass[aspectratio=43]{beamer}

\usepackage[french]{babel}
\usepackage{caption}
\usepackage[T1]{fontenc}
\usepackage{amsmath, amsfonts, amssymb, mathtools}
\usepackage{stmaryrd}
\usepackage{fancyhdr}
\usepackage{lipsum}
\usepackage{graphicx}
\usepackage[ddmmyyyy]{datetime}
\usepackage{adjustbox}
\usepackage[explicit]{titlesec}
\usepackage{pdfpages}
\usepackage{tikz}
\usepackage{pifont}
\usepackage{fontawesome5}

\usetheme{Madrid}

\definecolor{white}{gray}{0.98}
\definecolor{green}{HTML}{92D050}
\usefonttheme[onlymath]{serif}

\makeatletter
\definecolor{beamer@darkblue}{HTML}{3437eb} % changed this


\title[AutomateCellulaire]{TIPE - Automates cellullaires continus}
\subtitle{Et son application \`a la mod\'elisation de vie artificielle}
\author{Ars\`ene MALLET}
\date[\today]{\today}
\institute{Candidat - 22669}

\begin{document}


\begin{frame}
    \titlepage
\end{frame}

\begin{frame}
    \frametitle{Introduction}

    \begin{itemize}
        \setlength\itemsep{2em}
        \item Jeu a z\'ero-joueur : d\'etermin\'e par les
          conditions initiales
        \item "Jeu de la vie" par John Conway -> espace discret
        \item G\'en\'eralisation dans l'espace continu -> Lenia
    \end{itemize}
\end{frame}

\begin{frame}
    \frametitle{Automate Cellulaire (1)}

    \begin{alertblock}{D\'efinition - Noyau de convolution}
    
    \end{alertblock}

    \begin{alertblock}{D\'efinition - Fonction de Croissance}
    
    \end{alertblock}

\end{frame}

\begin{frame}
    \frametitle{Automate Cellulaire (2)}

    \begin{alertblock}{D\'efinition - Automate Cellulaire}
      Un automate cellullaires $\mathcal{A}$ est caract\'eris\'e par un 5-uplet
      $()$
      \begin{itemize}
        \item 
      \end{itemize}
    
    \end{alertblock}

    \begin{alertblock}{D\'efinition -  
    \end{alertblock}

\end{frame}
%{
% \setbeamercolor{background canvas}{bg=}
% \foreach \n in{1,...,58}  {
% \includepdf[pages=\n]{img/code/allCodeLandscape_compressed.pdf}
%}
%}

\end{document}
